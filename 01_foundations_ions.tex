\section{יונים, גרדיאנטים ופוטנציאל הממברנה}

\subsection{מושגים בסיסיים}
\begin{itemize}
\item \textbf{יון} – אטום או מולקולה בעלי מטען חשמלי כתוצאה מאיבוד או קבלת אלקטרונים.
\item \textbf{קטיון} (\myen{cation}) – יון בעל מטען חיובי (כגון \Na, \Kion).
\item \textbf{אניון} (\myen{anion}) – יון בעל מטען שלילי (כגון \Cl).
\item \textbf{דיפוזיה} (\myen{diffusion}) – תנועה ספונטנית מריכוז גבוה לריכוז נמוך, ללא השקעת אנרגיה.
\item \textbf{כוח חשמלי} – משיכה או דחייה הפועלת על יון בהתאם למטענו ולמתח הממברנה.
\item \textbf{שיווי־משקל של יון} – מצב שבו אין זרימה \textbf{נטו} של אותו יון דרך הממברנה,
אף שייתכן מעבר דו־כיווני מתמיד.
\end{itemize}

\subsection{שני הכוחות הפועלים על יונים}
על כל יון הפועל בסביבת ממברנה תאית פועלים תמיד שני כוחות:
\begin{itemize}
\item \textbf{הכוח הכימי (גרדיאנט ריכוזים)} – תלוי אך ורק בהפרש הריכוזים של היון בין פנים התא לחוץ התא.
\item \textbf{הכוח החשמלי} – תלוי במטען היון ובמתח החשמלי על פני הממברנה.
\end{itemize}

כיוון ועוצמת הזרימה בפועל נקבעים על־ידי \textbf{השילוב של שני הכוחות יחד},
המכונה \textbf{הגרדיאנט האלקטרוכימי}.

\subsection{הנחה חשובה במצב מנוחה}
במצב מנוחה של הנוירון:
\[
V_m \approx -70\,\text{mV}
\]
כלומר, פנים התא שלילי יחסית לסביבה החוץ־תאית.

מכאן נובע:
\begin{itemize}
\item קטיונים נמשכים פנימה ע"י הכוח החשמלי.
\item אניונים נדחים החוצה ע"י הכוח החשמלי.
\end{itemize}

\subsection{הרכב יוני כללי - \myen{''Salt Water Outside''}}
באופן אינטואיטיבי ניתן לזכור כי:
\textbf{הנוזל החוץ־תאי הוא ``מלוח''}. [cite: 32, 44]

משמעות הדבר: [cite: 31, 33]
\begin{itemize}
    \item ריכוז \myen{Na$^+$} ו-\myen{Cl$^-$} גבוה יותר מחוץ לתא. [cite: 31, 38, 41]
    \item ריכוז \myen{K$^+$} גבוה יותר בתוך התא. [cite: 31, 36, 43]
\end{itemize}

הרכב זה הוא הבסיס לגרדיאנטים הכימיים של היונים,
ולערכי פוטנציאל שיווי־המשקל שלהם.

\subsection{טבלת סיכום: ריכוזים, כוחות ו-\texorpdfstring{\(E_{\text{ion}}\)}{Eion}}

\begin{table}[H]
\centering
\renewcommand{\arraystretch}{1.35}
\begin{tabular}{|c|c|c|c|c|c|}
\hline
\textbf{יון} &
\textbf{מטען} &
\textbf{ריכוז גבוה} &
\textbf{כוח כימי} &
\textbf{כוח חשמלי במנוחה} &
\textbf{\(E_{\text{ion}}\)} \\
\hline
\Na & חיובי & חוץ־תאי & פנימה & פנימה & \(+55\,\text{mV}\) \\
\hline
\Kion & חיובי & תוך־תאי & החוצה & פנימה & \(-80\,\text{mV}\) \\
\hline
\Cl & שלילי & חוץ־תאי & פנימה & החוצה & \(-70\,\text{mV}\) \\
\hline
\end{tabular}
\caption{כיווני הכוחות וערכי שיווי־משקל של יונים עיקריים בנוירון.}
\end{table}

\subsection{הגרדיאנט האלקטרוכימי}
הגרדיאנט האלקטרוכימי הוא \textbf{הכוח הכולל} הפועל על יון,
והוא שילוב של:
\begin{itemize}
\item הגרדיאנט הכימי (ריכוזים)
\item הגרדיאנט החשמלי (מתח ומטען)
\end{itemize}

כאשר שני הכוחות פועלים באותו כיוון – הזרימה חזקה.  
כאשר הם פועלים בכיוונים מנוגדים – ייתכן שיווי־משקל.

\subsection{משוואת נרנסט – פוטנציאל שיווי־משקל של יון}
משוואת \textbf{נרנסט} מחשבת את פוטנציאל שיווי־המשקל של \textbf{יון יחיד}:

\[
E_{\text{ion}} = \frac{RT}{zF}
\ln\!\left(
\frac{[\text{ion}]_{\text{out}}}{[\text{ion}]_{\text{in}}}
\right)
\]

זהו המתח שבו הכוח הכימי והכוח החשמלי מאזנים זה את זה,
ולכן אין זרימה נטו של היון.

\subsubsection{פירוש הסימנים}
\begin{itemize}
\item \(E_{\text{ion}}\) – פוטנציאל שיווי־המשקל של היון.
\item \(R\) – קבוע הגזים, \(T\) – טמפרטורה מוחלטת, \(F\) – קבוע פאראדיי.
\item \(z\) – מטען היון (\(+1\) ל-\Na ול-\Kion, \(-1\) ל-\Cl).
\item \([\text{ion}]_{\text{out/in}}\) – ריכוז היון מחוץ ובתוך התא.
\end{itemize}

\subsubsection{קשר בין נרנסט למתח הממברנה}
\begin{itemize}
\item אם \myen{$V_m = E_{\text{ion}}$} – אין זרימה נטו של היון.
\item אם \myen{$V_m \neq E_{\text{ion}}$} – קיים דחף אלקטרוכימי לזרימה,
בתנאי שקיימת תעלה פתוחה.
\end{itemize}

\subsection{חדירות הממברנה (Permeability)}
לא כל היונים חוצים את הממברנה באותה מידה.
\textbf{חדירות} מתארת את מספר וסוג התעלות הפתוחות עבור כל יון.

במצב מנוחה:
\[
P_K \gg P_{Na} > P_{Cl}
\]

משמעות:
\begin{itemize}
\item קיימות הרבה יותר תעלות פתוחות ל-\Kion.
\item לכן \myen{$K^+$} הוא היון הדומיננטי בקביעת פוטנציאל המנוחה.
\item כתוצאה מכך:
\[
V_m \approx E_K
\]
\end{itemize}

\subsection{משאבת נתרן--אשלגן \myen{(\texorpdfstring{Na$^+$/K$^+$-ATPase}{Na+/K+-ATPase})}}
משאבת נתרן--אשלגן היא משאבה אקטיבית המשתמשת ב-\myen{ATP}
כדי לשמור על מפלי הריכוזים של \Na ו-\Kion.

בכל מחזור פעולה:
\begin{itemize}
\item מוציאה \textbf{3 יוני \Na} אל מחוץ לתא.
\item מכניסה \textbf{2 יוני \Kion} אל תוך התא.
\end{itemize}

מאפיינים חשובים:
\begin{itemize}
\item המשאבה פועלת \textbf{נגד מפל הריכוזים}.
\item היא \textbf{אלקטרוגנית} – מוציאה יותר מטען חיובי משהיא מכניסה.
\item תרומתה למתח הממברנה קטנה אך קיימת.
\end{itemize}

תפקידה העיקרי:
\begin{itemize}
\item שמירה על מפלי הריכוזים שעליהם מבוססים נרנסט ו-GHK.
\item אפשרות לקיום מתמשך של פוטנציאלי פעולה.
\end{itemize}

\textbf{שורת מבחן:}  
המשאבה אינה יוצרת פוטנציאל פעולה,
אלא מאפשרת למערכת החשמלית לפעול לאורך זמן.

\subsection{יונים ומולקולות שאינם חוצים את הממברנה}
הממברנה אינה חדירה ליונים גדולים ולחלבונים טעונים שלילית (\myen{A$^-$}),
הכלואים בתוך התא.

חלבונים אלו:
\begin{itemize}
\item תורמים לשליליות הפנימית של התא.
\item משפיעים בעקיפין על פיזור שאר היונים.
\end{itemize}

\subsection{חיבור למשוואת GHK}
מתח הממברנה בפועל (\myen{$V_m$}) נקבע ע"י:
\begin{itemize}
\item מספר יונים יחד (\Na, \Kion, \Cl)
\item הגרדיאנטים האלקטרוכימיים שלהם
\item \textbf{החדירות היחסית} של הממברנה לכל יון
\end{itemize}

לכן משתמשים במשוואת \textbf{Goldman--Hodgkin--Katz},
ולא במשוואת נרנסט בלבד.

\subsection{שורה למבחן}
\textbf{נרנסט} – יון אחד.  
\textbf{GHK} – כל היונים יחד + חדירות.  
\textbf{פוטנציאל פעולה} – שינוי דינמי בזמן של \myen{$V_m$}.

\subsection{משוואת Goldman--Hodgkin--Katz (GHK)}

משוואת \textbf{GHK} מחשבת את \textbf{פוטנציאל הממברנה בפועל} (\myen{$V_m$})
כאשר הממברנה חדירה למספר יונים במקביל,
וכל יון נשקל לפי \textbf{החדירות היחסית} שלו.

\[
V_m =
\frac{RT}{F}
\ln\!\left(
\frac{
P_{Na}[Na^+]_{\text{out}} +
P_{K}[K^+]_{\text{out}} +
P_{Cl}[Cl^-]_{\text{in}}
}{
P_{Na}[Na^+]_{\text{in}} +
P_{K}[K^+]_{\text{in}} +
P_{Cl}[Cl^-]_{\text{out}}
}
\right)
\]

\subsubsection{פירוש רכיבי המשוואה}
\begin{itemize}
\item \(V_m\) – פוטנציאל הממברנה.
\item \(R\) – קבוע הגזים.
\item \(T\) – טמפרטורה מוחלטת.
\item \(F\) – קבוע פאראדיי.
\item \(P_{ion}\) – חדירות הממברנה ליון מסוים.
\item \([\text{ion}]_{\text{out/in}}\) – ריכוז היון מחוץ ובתוך התא.
\end{itemize}

\textbf{מדוע \Cl מופיע הפוך?}  
מכיוון ש-\Cl הוא יון שלילי,
וכיוון הכוח החשמלי הפועל עליו הפוך.

\textbf{שורת מבחן:}  
נרנסט מחשבת מה קורה עם יון אחד בלבד.  
GHK מחשבת מה קורה בפועל בתא, עם כל היונים יחד.