% ==================================================
\section{נוירואנטומיה בסיסית: מנחים, חתכים ומבנים מרכזיים}
% ==================================================

\subsection{CNS ו-PNS}
\begin{itemize}
\item \textbf{\myen{CNS}} – מערכת העצבים המרכזית: מוח וחוט השדרה.
\item \textbf{\myen{PNS}} – מערכת העצבים ההיקפית: כל העצבים שמחוץ ל-CNS.
\end{itemize}


\subsection{למה מנחים וחתכים חשובים}
תמונות מוח (MRI, חתכים, איורים) תמיד מתוארות ביחס למנחים וחתכים.
טעות בזיהוי כיוון או חתך \ra טעות בפרשנות השאלה.


\subsection{מנחים אנטומיים (Directional terms)}
\begin{itemize}
\item \textbf{\myen{Anterior / Rostral}} – קדמי, לכיוון האף.
\item \textbf{\myen{Posterior / Caudal}} – אחורי, לכיוון העורף.
\item \textbf{\myen{Dorsal}} – לכיוון הגב; במוח האנושי זה בקירוב \textbf{למעלה}.
\item \textbf{\myen{Ventral}} – לכיוון הבטן; במוח האנושי זה בקירוב \textbf{למטה}.
\item \textbf{\myen{Medial}} – קרוב לקו האמצע (\myen{midline}).
\item \textbf{\myen{Lateral}} – רחוק מקו האמצע (לכיוון הצדדים).
\item \textbf{\myen{Superior}} – עליון.
\item \textbf{\myen{Inferior}} – תחתון.
\end{itemize}


\subsection{מבטים על המוח (Views)}
\begin{itemize}
\item \textbf{\myen{Lateral view}} – מבט מהצד (רואים המיספרה אחת).
\item \textbf{\myen{Medial view}} – מבט פנימי לאחר חצייה בין ההמיספרות.
\item \textbf{\myen{Superior view}} – מבט מלמעלה.
\item \textbf{\myen{Inferior view}} – מבט מלמטה (בסיס המוח).
\end{itemize}


\subsection{חתכים במוח (Planes / Sections)}
\begin{itemize}
\item \textbf{\myen{Sagittal}} – חתך ימין–שמאל.
\begin{itemize}
\item \myen{Mid-sagittal}: חתך על קו האמצע \ra רואים מבנים מדיאליים.
\end{itemize}

\item \textbf{\myen{Coronal / Frontal}} – חתך קדמי–אחורי.  
בדרך כלל רואים שתי המיספרות יחד (“פרוסות פנים”).

\item \textbf{\myen{Horizontal / Axial}} – חתך עליון–תחתון.  
נראה כמו פרוסות מלמעלה למטה.
\end{itemize}

\begin{figure}[H]
\centering
\includegraphics[width=0.5\linewidth]{brain_slices}
\caption{חתכי מוח: \myen{Sagittal}, \myen{Coronal}, \myen{Horizontal}}
\end{figure}

\subsection{מבנים מרכזיים: תלמוס מול צרבלום (“האגוזים”)}
\begin{itemize}
\item \textbf{תלמוס} (\myen{Thalamus}):  
מבנה עמוק ומרכזי במוח. משמש כתחנת ממסר עיקרית למידע סנסורי לקורטקס.

\item \textbf{צרבלום} (\myen{Cerebellum}):  
נמצא מאחור ולמטה (“מוח קטן”), בעל קפלים צפופים.  
אחראי על תיאום תנועה, דיוק ותזמון (לא יוזם תנועה).
\end{itemize}


\subsection{טעויות נפוצות במבחן}
\begin{itemize}
\item בלבול בין \myen{medial} ל-\myen{lateral}.
\item בלבול בין \myen{dorsal/ventral} לבין \myen{superior/inferior}.
\item זיהוי חתך \myen{coronal} כ-\myen{sagittal}.  
רמז: בקורונלי לרוב רואים שתי המיספרות יחד.
\end{itemize}