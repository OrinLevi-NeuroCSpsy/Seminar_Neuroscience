% ==================================================
\section{נוירואנטומיה: מנחים, חתכים ומבנים מרכזיים}
% ==================================================

\subsection{מנחים אנטומיים (Directional Terms)}
המונחים מתארים מיקום יחסי במוח.

\begin{itemize}
\item \textbf{\myen{Anterior / Rostral}} – קדמי, לכיוון האף.
\item \textbf{\myen{Posterior / Caudal}} – אחורי, לכיוון העורף.
\item \textbf{\myen{Dorsal}} – עליון (במוח האנושי).
\item \textbf{\myen{Ventral}} – תחתון.
\item \textbf{\myen{Medial}} – קרוב לקו האמצע.
\item \textbf{\myen{Lateral}} – רחוק מקו האמצע.
\end{itemize}

\textbf{שגיאה נפוצה במבחן:}  
לבלבל בין \myen{dorsal/ventral} לבין \myen{superior/inferior}.


\subsection{מבטים חיצוניים על המוח (\myen{Views})}
\begin{itemize}
\item \textbf{\myen{Superior view}} – מבט מלמעלה.
\item \textbf{\myen{Inferior view}} – מבט מלמטה (בסיס המוח).
\item \textbf{\myen{Lateral view}} – מבט מהצד.
\item \textbf{\myen{Medial view}} – מבט פנימי לאחר חצייה בין ההמיספרות.
\end{itemize}


\subsection{חתכים במוח (\myen{Sections / Planes})}
\begin{itemize}
\item \textbf{\myen{Coronal / Frontal}} – חתך קדמי–אחורי (רואים שתי המיספרות כמו “פנים”).
\item \textbf{\myen{Sagittal}} – חתך ימין–שמאל.
\item \textbf{\myen{Mid-sagittal}} – חתך בקו האמצע (רואים תלמוס, גזע מוח).
\item \textbf{\myen{Horizontal / Axial}} – חתך עליון–תחתון (“פרוסות”).
\end{itemize}

\textbf{טיפ למבחן:}  
אם רואים מבנים מדיאליים – זה כמעט תמיד \myen{mid-sagittal}.


\subsection{המיספרות וקישוריות}
\begin{itemize}
\item \textbf{\myen{Contralateral}} – קלט/פלט מהצד הנגדי.
\item \textbf{\myen{Ipsilateral}} – אותו צד.
\item כל המיספרה שולטת בעיקר על הצד הנגדי של הגוף.
\end{itemize}


\subsection{המוח הקטן – \myen{Cerebellum}}
\begin{itemize}
\item נמצא \textbf{מאחור ומתחת} לאונות האוקסיפיטליות.
\item נראה כמו “אגוז”.
\item אחראי על תיאום תנועה, דיוק ותזמון.
\item אינו יוזם תנועה אלא משפר אותה.
\end{itemize}

\textbf{שאלה נפוצה:}  
הצרבלום ≠ גזע המוח.


\subsection{גזע המוח (\myen{Brainstem})}
סדר אנטומי מלמטה למעלה:
\[
\text{\myen{Spinal cord}} \rightarrow
\text{\myen{Medulla}} \rightarrow
\text{\myen{Pons}} \rightarrow
\text{\myen{Midbrain}}
\]

\begin{itemize}
\item \textbf{Medulla}: נשימה, קצב לב, לחץ דם.
\item \textbf{Pons}: ויסות נשימה, שינה, חיבור לצרבלום.
\item \textbf{Midbrain}: רפלקסים חזותיים/שמיעתיים ובקרת תנועה בסיסית.
\end{itemize}

\textbf{קריטי למבחן:} פגיעה ב־\myen{Medulla} עלולה להיות קטלנית.


\subsection{קליפת המוח: \myen{Gyri} ו-\myen{Sulci}}
\begin{itemize}
\item \textbf{\myen{Gyrus}} – קפל.
\item \textbf{\myen{Sulcus}} – חריץ.
\item במהלך האבולוציה: יותר קפלים \ra יותר שטח קורטיקלי.
\end{itemize}


\subsection{חדרי המוח ו-CSF}
\begin{itemize}
\item 4 חדרים: שני לטרליים, שלישי, רביעי.
\item \myen{CSF} נוצר בחדרים.
\item תפקיד: הגנה, בלימת זעזועים.
\item לא רואים חדרים במבט חיצוני.
\end{itemize}


\subsection{קרומי המוח (\myen{Meninges})}
מהחוץ פנימה:
\begin{itemize}
\item \textbf{\myen{Dura mater}} – קשה וחיצונית.
\item \textbf{\myen{Arachnoid}} – קרום אמצעי.
\item \textbf{\myen{Pia mater}} – דק ונצמד למוח.
\end{itemize}

ה-\myen{CSF} נמצא ב-\myen{Subarachnoid space}.