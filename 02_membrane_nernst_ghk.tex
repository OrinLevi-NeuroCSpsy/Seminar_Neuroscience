\section{פוטנציאלים מדורגים, פוטנציאל פעולה ואינטגרציה סינפטית}

\subsection{שלושה סוגי פוטנציאלים}
בנוירון קיימים שלושה סוגי שינויים במתח הממברנה:
\begin{itemize}
\item \textbf{פוטנציאל מנוחה} (\myen{Resting potential})
\item \textbf{פוטנציאל מדורג} (\myen{Graded potential})
\item \textbf{פוטנציאל פעולה} (\myen{Action Potential, AP})
\end{itemize}

\subsection{פוטנציאל מדורג (\myen{Graded potential})}
פוטנציאל מדורג הוא שינוי \textbf{מקומי וזמני} ב-\myen{$V_m$}, הנגרם ע"י פתיחה של
\textbf{תעלות תלויות־ליגנד} (לרוב בעקבות נוירוטרנסמיטר).

\textbf{מאפיינים מרכזיים:}
\begin{itemize}
\item \textbf{מדורג} – עוצמת השינוי תלויה בעוצמת הגירוי.
\item \textbf{דועך עם מרחק} – אינו מתפשט לאורך כל האקסון.
\item יכול להיות:
\begin{itemize}
\item \textbf{דה־פולריזציה} – פחות שלילי (מעורר)
\item \textbf{היפר־פולריזציה} – יותר שלילי (מעכב)
\end{itemize}
\end{itemize}

\textbf{חשוב למבחן:}  
פוטנציאל מדורג \textbf{לא} obeys חוק “הכול או כלום”.

\subsection{EPSP ו-IPSP}
\begin{itemize}
\item \textbf{EPSP} (\myen{Excitatory Postsynaptic Potential}) –  
דה־פולריזציה פוסט־סינפטית, מקרבת את התא לסף העירור.
\item \textbf{IPSP} (\myen{Inhibitory Postsynaptic Potential}) –  
היפר־פולריזציה פוסט־סינפטית, מרחיקה את התא מסף העירור.
\end{itemize}

\textbf{דוגמאות יוניות נפוצות:}
\begin{itemize}
\item \myen{EPSP}: פתיחת תעלות \myen{Na$^+$} (לרוב דרך רצפטורים לגלוטמט).
\item \myen{IPSP}: פתיחת תעלות \myen{Cl$^-$} (\myen{GABA\textsubscript{A}}) או \myen{K$^+$} (\myen{GABA\textsubscript{B}}).
\end{itemize}

\subsection{GABA – עיכוב סינפטי: GABA\textsubscript{A} ו-GABA\textsubscript{B}}

\textbf{GABA} (\myen{Gamma-Aminobutyric Acid}) הוא הנוירוטרנסמיטר
\textbf{המעכב הראשי} במערכת העצבים המרכזית.

ל-GABA שני סוגי רצפטורים עיקריים, בעלי מנגנון שונה אך תוצאה משותפת:
\textbf{הקטנת הסיכוי ליצירת פוטנציאל פעולה}.

\subsubsection{השוואה בין רצפטורי GABA}

\begin{table}[H]
\centering
\renewcommand{\arraystretch}{1.35}
\begin{tabular}{|c|c|c|c|c|}
\hline
\textbf{רצפטור} &
\textbf{סוג רצפטור} &
\textbf{תעלה} &
\textbf{מהירות} &
\textbf{מנגנון עיכוב} \\
\hline
GABA\textsubscript{A} &
יונוטרופי &
Cl$^-$ &
מהיר &
Shunting / היפר־פולריזציה \\
\hline
GABA\textsubscript{B} &
מטבוטרופי (GPCR) &
K$^+$ (בעקיפין) &
איטי &
היפר־פולריזציה \\
\hline
\end{tabular}
\caption{השוואה בין סוגי רצפטורי GABA והאופן שבו הם יוצרים IPSP.}
\end{table}

\subsubsection{GABA\textsubscript{A} – תעלות כלור (Cl$^-$)}
קשירת GABA ל-\myen{GABA\textsubscript{A}} פותחת \textbf{תעלות Cl$^-$}
ויוצרת \textbf{IPSP מהיר}.

חשוב:
\begin{itemize}
\item גם אם \(E_{Cl} \approx V_m\), פתיחת תעלות Cl$^-$
\textbf{עדיין מעכבת}.
\item הסיבה: \textbf{Shunting inhibition} –  
עלייה במוליכות הממברנה ש״מנקזת״ EPSP-ים נכנסים.
\end{itemize}

כלומר, העיכוב אינו חייב להיות שינוי גדול במתח,
אלא \textbf{הקטנת ההשפעה של גירויים מעוררים}.

\subsubsection{GABA\textsubscript{B} – תעלות אשלגן (K$^+$)}
\myen{GABA\textsubscript{B}} הוא רצפטור מטבוטרופי הפועל דרך שליחים שניוניים.

קשירת GABA גורמת ל:
\begin{itemize}
\item פתיחה עקיפה של תעלות K$^+$.
\item יציאת K$^+$ מהתא.
\item \textbf{היפר־פולריזציה ברורה} של הממברנה.
\end{itemize}

מאפיינים:
\begin{itemize}
\item איטי יותר מ-GABA\textsubscript{A}.
\item ממושך יותר.
\item מרחיק את התא מסף העירור.
\end{itemize}

\subsubsection{שורה למבחן}
\textbf{GABA\textsubscript{A}} – עיכוב מהיר דרך Cl$^-$ (Shunting).  
\textbf{GABA\textsubscript{B}} – עיכוב איטי דרך K$^+$.  
שניהם יוצרים \textbf{IPSP} ומקטינים את הסיכוי לפוטנציאל פעולה.

\subsection{סכימה עצבית (\myen{Summation})}
פוטנציאלים מדורגים יכולים להצטבר:
\begin{itemize}
\item \textbf{סכימה בזמן} (\myen{Temporal summation}) – גירויים חוזרים מאותה סינפסה.
\item \textbf{סכימה במרחב} (\myen{Spatial summation}) – גירויים מכמה סינפסות שונות.
\end{itemize}

ההחלטה האם ייווצר פוטנציאל פעולה מתקבלת ב-\textbf{axon hillock}.

\subsection{פוטנציאל פעולה (\myen{Action Potential})}
פוטנציאל פעולה הוא שינוי חד ומהיר ב-\myen{$V_m$}, המתפשט לאורך האקסון.

\textbf{מאפיינים מרכזיים:}
\begin{itemize}
\item obeys חוק \textbf{“הכול או כלום”}.
\item גודל קבוע – אינו תלוי בעוצמת הגירוי.
\item מתפשט ללא דעיכה.
\end{itemize}

\subsection{סף העירור (\myen{Threshold})}
\begin{itemize}
\item סף עירור טיפוסי: \(-55\,\text{mV}\).
\item חציית הסף גורמת לפתיחה מאסיבית של תעלות \Na תלויות־מתח.
\item נוצר \textbf{משוב חיובי} → דה־פולריזציה מהירה.
\end{itemize}

\subsection{שלבי פוטנציאל הפעולה}
\begin{enumerate}
\item \textbf{דה־פולריזציה}: פתיחת תעלות \Na תלויות־מתח.
\item \textbf{רפולריזציה}: אינאקטיבציה של \Na ופתיחת תעלות \Kion.
\item \textbf{היפר־פולריזציה}: יציאה עודפת של \Kion.
\end{enumerate}

\subsection{התקופה הרפרקטורית}
לאחר פוטנציאל פעולה קיימת תקופה שבה קשה או בלתי אפשרי לייצר פוטנציאל נוסף:
\begin{itemize}
\item \textbf{תקופה רפרקטורית מוחלטת} –  
לא ניתן לייצר AP נוסף (תעלות \Na באינאקטיבציה).
\item \textbf{תקופה רפרקטורית יחסית} –  
אפשרי AP, אך נדרש גירוי חזק במיוחד (היפר־פולריזציה).
\end{itemize}

\subsection{למה נוירוטרנסמיטר לא יוצר פוטנציאל פעולה ישירות}
\begin{itemize}
\item נוירוטרנסמיטר פותח \textbf{תעלות תלויות־ליגנד}.
\item פתיחה זו יוצרת \textbf{פוטנציאל מדורג בלבד}.
\item פוטנציאל פעולה דורש פתיחה של \textbf{תעלות \Na תלויות־מתח}.
\item רק כאשר סכימת ה-EPSP מביאה את \myen{$V_m$} לסף – נוצר AP.
\end{itemize}

\textbf{שורת מבחן:}  
נוירוטרנסמיטר \textbf{מתחיל} את האות –  
תעלות תלויות־מתח \textbf{יוצרות} את פוטנציאל הפעולה.

\subsection{קידוד עוצמה}
\begin{itemize}
\item עוצמת הגירוי \textbf{לא} מקודדת בגודל ה-AP.
\item העוצמה מקודדת ב-\textbf{תדירות} פוטנציאלי הפעולה (\myen{Rate coding}).
\end{itemize}